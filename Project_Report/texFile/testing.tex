\documentclass{standalone}

\begin{document}

\section*{6.1\hspace{0.3 cm} Unit Testing}
\addcontentsline{toc}{section}{6.1\hspace{0.3 cm} Unit Testing} \vspace{0.5cm}

\hspace{0.5cm} In unit testing where we check the individual units and components that we used in frontend are tested properly. The purpose is to validate the each unit of code performs as expected earliar. The bug-prevention objective is superior to others and implies not only anticipation but also prevention of defects from recurring in the future.\vspace{1cm}

\section*{6.2\hspace{0.3 cm} Integration Testing}
\addcontentsline{toc}{section}{6.2\hspace{0.3 cm} Integration Testing} \vspace{0.3cm}

\hspace{0.5cm} Also In integration testing we check the hardware and software environment that are intregrated or not. This types of testing helps us the software's compliance and with its specified requirement. It is performed to verify the interactions between the modules of this project.\vspace{1cm}

\section*{6.3\hspace{0.3 cm} System Testing}
\addcontentsline{toc}{section}{6.3\hspace{0.3 cm} System Testing} \vspace{0.3cm}

\hspace{0.5cm}System Testing is carried out on the whole system in the context of either system requirement specifications or functional requirement specifications or in the context of both. This system testing is basically performed by a testing team that is independent to us(development team) and that helps the actual test of the quality. They also detect some bugs before users find them. After testing we sorted out of these bugs and worked properly later.\vspace{0.7cm}

\begin{enumerate}

\item Testing the fully integrated applications including external peripherals in order to check how components interact with one another and with the system as a whole.
\item Verify thorough testing of every input in the application to check for desired outputs.
\item Testing of the user's experience with the application. \vspace{0.7cm}

\end{enumerate}
\section*{6.4\hspace{0.3 cm} User Acceptance Testing}
\addcontentsline{toc}{section}{6.4\hspace{0.3 cm} User Acceptance Testing} \vspace{0.3cm}

\hspace{0.5cm} QA team make sure that the product satisfies the user requirements and works as desired. In the process of the software verification and validation at these stage we hire a team(testing team) from different activities they are not dependent to us check the system that works perfectly or not, that satisfies our users(students) as expected. These testing also helps us to prevent software bug and shorten the product time to our users

\end{document}